% Template for Cogsci submission with R Markdown

% Stuff changed from original Markdown PLOS Template
\documentclass[10pt, letterpaper]{article}

\usepackage{cogsci}
\usepackage{pslatex}
\usepackage{float}
\usepackage{caption}

% amsmath package, useful for mathematical formulas
\usepackage{amsmath}

% amssymb package, useful for mathematical symbols
\usepackage{amssymb}

% hyperref package, useful for hyperlinks
\usepackage{hyperref}

% graphicx package, useful for including eps and pdf graphics
% include graphics with the command \includegraphics
\usepackage{graphicx}

% Sweave(-like)
\usepackage{fancyvrb}
\DefineVerbatimEnvironment{Sinput}{Verbatim}{fontshape=sl}
\DefineVerbatimEnvironment{Soutput}{Verbatim}{}
\DefineVerbatimEnvironment{Scode}{Verbatim}{fontshape=sl}
\newenvironment{Schunk}{}{}
\DefineVerbatimEnvironment{Code}{Verbatim}{}
\DefineVerbatimEnvironment{CodeInput}{Verbatim}{fontshape=sl}
\DefineVerbatimEnvironment{CodeOutput}{Verbatim}{}
\newenvironment{CodeChunk}{}{}

% cite package, to clean up citations in the main text. Do not remove.
\usepackage{apacite}

% KM added 1/4/18 to allow control of blind submission
\cogscifinalcopy

\usepackage{color}

% Use doublespacing - comment out for single spacing
%\usepackage{setspace}
%\doublespacing


% % Text layout
% \topmargin 0.0cm
% \oddsidemargin 0.5cm
% \evensidemargin 0.5cm
% \textwidth 16cm
% \textheight 21cm

\title{Active, associative, or both?: Preschool children reason about
the acoustic environment for third-party goal optimization}



\newlength{\cslhangindent}
\setlength{\cslhangindent}{1.5em}
\newenvironment{CSLReferences}%
  {}%
  {\par}

\begin{document}

\maketitle

\begin{abstract}
Despite the unpredictible and ubiquitous nature of noise in the natural
environment, most children still manage to extract the linguistic,
cognitive, and social skills needed to engage typically with the world
around them. This is no small feat; it is still largely unknown what
strategies children use to extract information from their environment in
the presence of noise. One possbility is that children have learned how
to reason about their acoustic context to optimize goals and goal
outcomes. In Experiment 1, we presented preschool children with a set of
auditory stimuli, meant to approximate various acoustic environments,
and activity goals to complete within those environments. Results showed
that children reliably integrated auditory information with a
third-party's changing goals to optimize another person's outcomes.
Experiment 2 built on this framework by replacing familiar activity
goals with novel ones to assess the underlying mechanism driving
children's decision-making. This time, children were still integrating
across auditory information and activity goals, but did so less
reliably. This suggests that {[}in progress{]}.

\textbf{Keywords:}
active learning; associative learning; auditory noise; cognitive
development; decision making
\end{abstract}

\hypertarget{introduction}{%
\section{Introduction}\label{introduction}}

Children are excavators; they routinely build linguistic, cognitive,
social, and emotional skills through interacting with their
environments. They attend to the statistical regularities of linguistic
natural speech streams, but can even do so with artificial ones
(Saffran, Aslin, \& Newport, 1996). They can exploit the emotional
expressions of others to determine whether a novel object is worth
exploration, thereby maximizing efficiency (Wu \& Gweon, 2021). In the
face of unfamiliar others, they can evaluate their caregivers'
interactions to determine whether these unfamiliar others might be
potential social partners (Thomas, Saxe, \& Spelke, 2022). And when a
speaker offers new information about novel words, preschool children can
rapidly update their predictions about what the speaker will say next
(Havron, Carvalho, Fiévet, \& Christophe, 2019).

However, several accounts of children's early auditory environments
{[}or of auditory environments in general{]}, including skills like
language processing, omit the presence of noise in the channel (Gibson,
Bergen, \& Piantadosi, 2013). Noise, any stimulus that is unwanted or
unattended to, can be internal or external to the system. Models that
have acknowledged noise as a meaningful element to the output have
reasonably focused on internal noise, such as errors in production or
perception, or a limited linguistic repertoire (Gibson et al., 2013;
Shannon, 1948). In this account, however, we ask what strategies
preschool children use to extract information from their environment in
the presence of external noise exclusively.

Extracting information from the environment in noise is not a trivial
process; children are notably worse than adults at skills such as speech
perception and word recognition in noise (Bjorklund \& Harnishfeger,
1990; Klatte, Bergström, \& Lachmann, 2013), and exhibit real challenges
in word learning under background noise constraints (McMillan \&
Saffran, 2016). Because noise generally increases cognitive load during
certain attention and spatial tasks, children are less able to flexibly
adapt strategies to successfully complete these tasks than adults (Loh,
Fintor, Nolden, \& Fels, 2022). There is also emerging evidence that
high levels of sustained noise exposure changes the cortical thickness
of the left inferiopr frontal gyrus in infants (Simon, Merz, He, \&
Noble, 2022). Importantly, these effects of noise are not happening
exclusively at the unconscious level; even young children are
perceptually aware of excessive noise exposure (McAllister, Rantala, \&
Jónsdóttir, 2019). In sum, noise is both influential and palpable, and
we aimed to explore how young children actively mitigate its effects.

Despite the overwhelming presence and potential deleterious consequences
of noise, most typically developing children successfully extract
relevant information from the acoustic environment. That you can read
this paper is evidence of this phenomenon. One entry point into how this
might be possible is in evaluating children's interactions with the
acoustic environment, a space that overlaps with active learning. The
active learning literature has historically explored children's
interaction with individual stimuli within the environment (e.g.,
Settles, 2009). For example, previous work has shown that preschool
children use active learning strategies to approach objects in a novel
task in order to optimize performance (Ruggeri, Swaboda, Sim, \& Gopnik,
2019). Even infants harness the utility of active learning by updating
their expectation about what could be learned from an object that
behaved unexpectedly, such as ball moving through a solid wall (Stahl \&
Feigenson, 2015). Additionally, infants as young as 7 months have been
shown to efficiently allocate their attention to visual stimuli that is
neither too complex nor too simple (Kidd, Piantadosi, \& Aslin, 2012).
This ability is also evidenced in language learning (Foushee,
Srinivasan, \& Xu, 2022). It is no longer the dominant view that
children are passively absorbing linguistic input from the acoustic
environment. Instead, they are constantly assessing optimal approaches
for language learning. For example, there is some agreement that
children can and do build linguistic prowess through overheard speech,
and that this coordinated skill seems to improve with age (Foushee,
Srinivasan, \& Xu, 2021). But children also adjust their attention to
linguistic stimuli such as grammar directed at them based on its
learnability (Gerken, Balcomb, \& Minton, 2011). Taken together, active
learning suggests that children make contact with features of their
environment, with varying degrees and sources of motivation, and that
they do this quite readily and consistently.

Beyond individual stimuli, it is also possible that children engage in
active learning through making decisions about the acoustic environment
itself for optimization. Importantly, an acoustic environment can be
flexibly defined, so we refer to it as the physical space in which the
child interacts both in and on, and one that moves with the child. In
this environment, children may adjust their engagement based on both
auditory information and previously defined goals. For example, a child
might choose to read or be read to in a library or in an empty room
because a quiet space best aligns with the goal of taking in a
storybook. We refer to this as environmental selection, the preferential
process of selecting environments that align with goal optimization.
Environmental selection is a type of active learning, one that may allow
children to exploit environmental variation to extract important
information under noisy auditory constraints. Indeed, we have previously
shown that adults, when given both a set of goals and variable auditory
environments, converged on pairings meant to optimize those goals
{[}retracted{]}. However, we were unable to show that preschool children
also engage in environmental selection beyond an emergent quality.

In the current paper, we reconfigured our paradigm to better understand
both whether and how preschool children use environmental selection for
goal optimization. In Experiment 1, we asked children to match a set of
goals to one or more auditory environments where appropriate. Then in
Experiment 2, we presented children with novel activities and asked them
to complete the same task to explore the conceptual boundaries of this
ability. Taken together, this set of experiments aims to expand our
understanding of children's adaptive strategies in compromised auditory
environments.

\hypertarget{experiment-1}{%
\section{Experiment 1}\label{experiment-1}}

In our first Experiment, we evaluated preschool children's environmental
selection, their integration of both auditory information and a third
party's goals, for familiar activities. We asked whether they would
differentially select environment-goal pairings that optimized another
person's goals. If children are systematically pairing based on
outcomes, this may suggest that they are, in fact, attuned to the
environment as a feature of goal optimization.

\hypertarget{methods}{%
\subsection{Methods}\label{methods}}

\hypertarget{participants}{%
\subsubsection{Participants}\label{participants}}

71 children (3;0 - 5;11 years, mean age = 4 years , 28.2\%
Caucasian/White) were recruited from either a local Bay Area nursery
school or children's museum. Participants were typically developing, had
normal or corrected-to-normal vision, and heard English at least 75\% of
the time at home. An additional 0 children were ultimately excluded from
analysis due to response bias (provided the same pattern of responses
for 100\% of trials), experimenter error, or exhibited severe lapses in
attention. Caregivers provided written consent while children provided
verbal assent before participation.

\hypertarget{materials-and-procedure}{%
\subsubsection{Materials and Procedure}\label{materials-and-procedure}}

\begin{CodeChunk}
\begin{figure}[t]

{\centering \includegraphics{figs/e3-stimuli-1} 

}

\caption[Experimental setup and stimuli]{Experimental setup and stimuli. Participants were shown four wooden houses, each with an associated sound [instrumental music, multi-talker babble, silence, and white noise], and a list of four activities [dance, read, sleep, talk] that two charactrs in the game wanted to complete. Participants determined whether the two characters should or should not complete an activity in each of the four houses. Responses were independent of each other.}\label{fig:e3-stimuli}
\end{figure}
\end{CodeChunk}

A trained undergraduate research assistant served as the experimenter
for the task. The experimenter first introduced participants to two
small plastic figures named Joe and Mandy and to four wooden houses with
a felt door on the front. The experimenter then showed participants a
list of four images, each depicting one activity Joe and Mandy wanted to
do together. The experimenter explained that when the door opened, each
house would either play a sound or it wouldn't play anything at all. Joe
and Mandy could choose whether or not to complete an activity in each
house, and their decisions would be entirely based on participants'
responses. Importantly, these decisions were independent of each other;
participants could decide to have Joe and Mandy complete the same
activity in more than one room if appropriate. A sound button was
attached to the back of each house and hidden from the participant's
view so when the experimenter opened the door, they also pressed down on
the button to play the appropriate sound. The wooden houses were lined
up on a table several inches apart with the participant seated facing
the door of the house and the experimenter on the opposite side facing
the sound buttons. Figure 1 illustrates the setup, the four activities,
and the four auditory stimuli.

The experimenter began the task with the first image on the list and
told participants, ``It looks like Joe and Mandy want to {[}sleep{]}.
Let's look at each room and see if Joe and Mandy should {[}sleep{]}
inside.'' The experimenter then opened the door to the first house
{[}the experimenter always began with the first house on their left/the
first house on the participant's right{]} and pressed down on the sound
button. At the end of the audio clip, the experimenter closed the door
and asked participants two questions. Participants only heard each audio
once per trial. The experimenter repeated this process for the three
remaining houses before moving on to the next activity. In total,
participants completed 16 trials- 4 trials for each activity times 4
trials for each auditory stimulus. Trials were counterbalanced such that
the presentation order was randomized into four conditions.

Each auditory stimulus was 7s in length and equalized to a root mean
square (RMS) of 65dB. The multi-talker babble was an overlay of five
adult native English speakers reading short, unrelated sentences
(Panfili (n.d.)). The white noise was engineered in Audacity. The
instrumental music contained no human speech. Both the activities and
auditory stimuli were selected based on a sample of adults run
previously. One possible confound is the auditory stimuli suggests
varying numbers of people inside the wooden houses. For example, the
house paired with multi-talker babble may appear to have more figures or
people inside than the houses paired with instrumental music, silence,
and white noise. This may inadvertently influence children's decisions
on whether or not a house is appropriate for a specific activity for
reasons other than the auditory stimuli. To address this, we opened the
top of each house and showed children that two other figures were
inside.

We asked participants two questions which served as our DVs: (1)
``Should Joe and Mandy {[}read/dance/sleep/talk{]} in this room?'' and
(2) ``Why did you say Joe and Mandy {[}should/shouldn't{]}
{[}read/dance/sleep/talk{]} in this room?''

\hypertarget{results-and-discussion}{%
\subsection{Results and Discussion}\label{results-and-discussion}}

\begin{CodeChunk}
\begin{figure}[t]

{\centering \includegraphics{figs/e3b-bar-1} 

}

\caption[Results from Experiment 1]{Results from Experiment 1. Participants' rating of the appropriateness of an auditory stimulus by activity. Individual bars correspond to one age bin of 3, 4, or 5. A rating score of 0 indicates a rejection of the pairing [Joe and Mandy should not complete a particular activity in this environment] while a score of 1 indicates an affirmation of the pairing [Joe and Mandy should complete a particular activity in this environment]. A 2-alternative forced choice design resulted in no preference at 50\%.  Error bars show 95\% confidence intervals.}\label{fig:e3b-bar}
\end{figure}
\end{CodeChunk}

If preschool children can reason about how the acoustic environment
might influence goal optimization, and can make decisions to this end,
we should expect participants to show clear preferences for activities
paired with particular auditory stimuli. We preregistered {[}redacted{]}
a Bayesian mixed-effects logistic regression to predict participants'
response as a function of auditory stimulus, activity, and age, with a
maximal random effect structure (random intercept by participant). In
this and subsequent models, we used the package default of weakly
informative priors (normal distributions on coefficients with SD=2.5,
scaled to predictor magnitudes).

On average, preschool children showed clear preferences for certain
environment-activity pairings (intercept: \(\beta\) = 4.67, 95\% Crl =
{[}1.71 - 7.67{]}). And although there was no aggregated effect of age
on preference (intercept: \(\beta\) = -0.5, 95\% Crl = {[}-1.22 -
0.21{]}), there were age effects on preference under 5-talker babble
(intercept: \(\beta\) = -1.61, 95\% Crl = {[}-2.62 - -0.64{]}), as well
as interactions between (1) the 5-talker babble-talk activity pairing
and age (intercept: \(\beta\) = 1.96, 95\% Crl = {[}0.64 - 3.25{]}) and
(2) the silence-sleep activity pairing and age (intercept: \(\beta\) =
1.56, 95\% Crl = {[}0.29 - 2.87{]}). Figure 2 summarizes this data.

These findings suggest that across the preschool years, children are
evaluating the acoustic environments to make decisions about third-party
goal optimization. We propose that this is evidence of basic
environmental selection, and that children as young as three can
reliably engage in it.

\hypertarget{experiment-2}{%
\section{Experiment 2}\label{experiment-2}}

In Experiment 1, we found that preschool children do engage in
environmental selection, such that they may make decisions about optimal
environments for goal selection based on auditory information. We also
found that this ability was generall stable across preschol years.
However, it is possible that children succeeded in this task not because
they were engaging in some cognitively flexible process, but that they
were relying on learned conventions. For example, children may have
paired napping with silence because they have learned that any sleeping
activity is done in quiet, and not because they recognize that silence
might be the most optimal auditory environment to both fall and stay
asleep.

Young children harness associative learning in a host of contexts. For
example, associative models of word learning suggest that children
exploit contextual information to acquire the meanings of novel words
(Sloutsky, Yim, Yao, \& Dennis, 2017). This is especially helpful given
referential uncertainty of unfamiliar words that co-occur with other
referents with varying individual probabilities (Quine \& Van, 1960).
Some of the most convincing associative models of word learning are
probabilistic at their core and suggest that word learning happens, in
part, because learners are constantly assessing the probabilities that a
word's meaning is associated with an unfamiliar word. Moreover, these
probabilities are updated as the lexicon expands (see Stevens, Gleitman,
Trueswell, \& Yang, 2017). But associative learning is not relegated to
language development; indeed, humans can develop associative links from
third-party social interactions (Thiele, Hepach, Michel, Gredebäck, \&
Haun, 2021), to remember sequences of stimuli without overloading our
cognitive system (Tosatto, Fagot, Nemeth, \& Rey, 2022), and even to
recognize potential environmental threats, such as snakes and spiders,
in infancy (Rakison, 2022). One possibility, then, is that environmental
selection is driven by associative learning for young children.

On the other hand, children may not be relying solely on learned
associations. Instead, there may exist an additional layer for
evaluation, that of active learning. Staunchly traditional associative
accounts view the learner as a passive agent, one who learns what is
true about their environment but does not necessarily act on their
environment. One account that combines associative learning with active
learning is Active Bayesian Associative Learning {[}ABAL{]}, and this
may be the driving mechanism behind environmental selection (see
Kruschke, 2008). In this context, ABAL suggests that humans can not only
make associations between their acoustic environment and a set of goals,
but they can also assign probabilities to a range of stimuli that they
have never actually encountered. This process is cognitively flexible
and requires some degree of active engagement with the environment.

To determine what might be driving children's environmental selection in
particular, we re-ran Experiment 1 on a new sample of preschool children
and replaced familiar activities with novel ones.

\hypertarget{methods-1}{%
\subsection{Methods}\label{methods-1}}

\hypertarget{participants-1}{%
\subsubsection{Participants}\label{participants-1}}

12 children (3;0 - 5;11 years, mean age = 4.08 years , 41.7\%
Caucasian/White) were recruited from either a local Bay Area nursery
school or children's museum. Participants were typically developing, had
normal or corrected-to-normal vision, and heard English at least 75\% of
the time at home. An additional 0 children were ultimately excluded from
analysis due to response bias (provided the same pattern of responses
for 100\% of trials), experimenter error, or exhibited severe lapses in
attention. Caregivers provided written consent while children provided
verbal assent before participation.

\hypertarget{materials-and-procedure-1}{%
\subsubsection{Materials and
Procedure}\label{materials-and-procedure-1}}

The procedures for Experiment 2 were nearly identical to Experiment 1
with one notable difference. To determine whether preschool children use
environmental selection flexibly to novel activities, we presented
participants with a new list of activities- (1) Fraw: when someone reads
you a bedtime story right before you fall asleep, (2) Gobb: when you are
looking for something to do because you are really bored, (3) Plip: when
you spin around in circles to the beat until you get really dizzy, and
(4) Terb: when you don't want anyone else to know your tummy is making
noise. We selected these novel activities based on an adult sample we
previously ran online, where we found these four activities elicited the
widest distribution of responses among participants.

We asked participants two questions which served as our DVs: (1)
``Should Joe and Mandy {[}fraw/gobb/plip/terb{]} in this room?'' and (2)
``Why did you say Joe and Mandy {[}should/shouldn't{]}
{[}fraw/gobb/plip/terb{]} in this room?''

\hypertarget{results-and-discussion-1}{%
\subsection{Results and Discussion}\label{results-and-discussion-1}}

\begin{CodeChunk}
\begin{figure}[t]

{\centering \includegraphics{figs/e4b-bar-1} 

}

\caption[Results from Experiment 2]{Results from Experiment 2. Participants' rating of the appropriateness of an auditory stimulus by activity. Individual bars correspond to one age bin of 3, 4, or 5. A rating score of 0 indicates a rejection of the pairing [Joe and Mandy should not complete a particular activity in this environment] while a score of 1 indicates an affirmation of the pairing [Joe and Mandy should complete a particular activity in this environment]. A 2-alternative forced choice design resulted in no preference at 50\%.}\label{fig:e4b-bar}
\end{figure}
\end{CodeChunk}

If preschool children rely solely on convention when evaluating the
acoustic environment, that is, they have acquired associative links
between familiar activities and their acoustic contexts, we should
expect that children will have no strong preferences for pairing novel
activities with any particular acoustic context. If, however, children
can reason flexibly about how the acoustic environment influences goal
optimization and outcomes, we should expect that children show clear
preferences for acoustic contexts even with activities they have never
actually encountered. We preregistered {[}retracted{]} the same Bayesian
mixed-effects logistic regression as in Experiment 1 predicting
environmental preference as a function of auditory stimuli, activity,
and age.

\hypertarget{general-discussion}{%
\section{General Discussion}\label{general-discussion}}

In this set of Experiments, we explored preschool children's reasoning
about their acoustic environments. To this end, we asked if children
engage in environmental selection for goal optimization. In Experiment
1, we found that across the preschool years, children can reliably
evaluate the acoustic environment to inform their decisions about
third-party goals. In Experiment 2, we asked what underlying mechanism
might be driving this ability by asking children to reason about novel
activities. Results showed that {[}in progress{]}

In some unpublished work where we asked adults to reason about the links
between a larger set of novel activities and acoustic environments than
were presented to children in Experiment 2, we found that adults
reliably paired these novel activities with particular acoustic
environments despite having never encountered these activities before
(intercept: \(\beta\) = 4.82, 95\% Crl = {[}4.23 - 5.38{]}).

This work is not without its limitations, which fuel our future
directions. In this set of experiments, we asked children to reason
about the goals of others; it is possible that children's preferences
for certain acoustic environments may vary if they were instead asked to
reason about their own goals. This work also does not directly test a
strategy children might use to extract information from the environment
under noise constraints. Instead, it lays the foundation for
understanding how children evaluate acoustic environments with varying
degrees of noise, and a possible mechanism that drives it. By exploring
children's environmental evaluations and their flexibility beyond
familiar associations, we might later manipulate children's own acoustic
environments to observe the utility of environmental selection in
action. We believe the current studies are critical interim steps to
this end because they will inform the direction of future research.

Environmental noise exposure is here to stay; noise pollution in the
United States affects everyone at some time or another, but some
evidence suggests that it disproportionately affects communities of
color and those of lower socioeconomic status, who tend to reside in
more densely populated regions (Casey et al., 2017). This could have
downstream consequences on linguistic and cognitive skills, as well as
on academic achievement. Future research should be sensitive to both the
acute and chronic effects of noise exposure on children, in particular,
and study strategies that can be implemented to ameliorate these
effects. We believe that environmental selection could be one such
strategy, and that by three years, children have the hardware to use it
effectively.

\hypertarget{references}{%
\section{References}\label{references}}

\setlength{\parindent}{-0.1in} 
\setlength{\leftskip}{0.125in}

\noindent

\hypertarget{refs}{}
\begin{CSLReferences}{1}{0}
\leavevmode\vadjust pre{\hypertarget{ref-bjorklund1990}{}}%
Bjorklund, D. F., \& Harnishfeger, K. K. (1990). The resources construct
in cognitive development: Diverse sources of evidence and a theory of
inefficient inhibition. \emph{Developmental Review}, \emph{10}(1),
48--71.

\leavevmode\vadjust pre{\hypertarget{ref-casey2017}{}}%
Casey, J. A., Morello-Frosch, R., Mennitt, D. J., Fristrup, K., Ogburn,
E. L., \& James, P. (2017). Race/ethnicity, socioeconomic status,
residential segregation, and spatial variation in noise exposure in the
contiguous united states. \emph{Environmental Health Perspectives},
\emph{125}(7), 077017.

\leavevmode\vadjust pre{\hypertarget{ref-foushee2021}{}}%
Foushee, R., Srinivasan, M., \& Xu, F. (2021). Self-directed learning by
preschoolers in a naturalistic overhearing context. \emph{Cognition},
\emph{206}, 104415.

\leavevmode\vadjust pre{\hypertarget{ref-foushee2022}{}}%
Foushee, R., Srinivasan, M., \& Xu, F. (2022). Active learning in
language development.

\leavevmode\vadjust pre{\hypertarget{ref-gerken2011}{}}%
Gerken, L., Balcomb, F. K., \& Minton, J. L. (2011). Infants avoid
`labouring in vain'by attending more to learnable than unlearnable
linguistic patterns. \emph{Developmental Science}, \emph{14}(5),
972--979.

\leavevmode\vadjust pre{\hypertarget{ref-gibson2013}{}}%
Gibson, E., Bergen, L., \& Piantadosi, S. T. (2013). Rational
integration of noisy evidence and prior semantic expectations in
sentence interpretation. \emph{Proceedings of the National Academy of
Sciences}, \emph{110}(20), 8051--8056.

\leavevmode\vadjust pre{\hypertarget{ref-havron2019}{}}%
Havron, N., Carvalho, A. de, Fiévet, A.-C., \& Christophe, A. (2019).
Three-to four-year-old children rapidly adapt their predictions and use
them to learn novel word meanings. \emph{Child Development},
\emph{90}(1), 82--90.

\leavevmode\vadjust pre{\hypertarget{ref-kidd2012}{}}%
Kidd, C., Piantadosi, S. T., \& Aslin, R. N. (2012). The goldilocks
effect: Human infants allocate attention to visual sequences that are
neither too simple nor too complex. \emph{PloS One}, \emph{7}(5),
e36399.

\leavevmode\vadjust pre{\hypertarget{ref-klatte2013}{}}%
Klatte, M., Bergström, K., \& Lachmann, T. (2013). Does noise affect
learning? A short review on noise effects on cognitive performance in
children. \emph{Frontiers in Psychology}, \emph{4}, 578.

\leavevmode\vadjust pre{\hypertarget{ref-kruschke2008}{}}%
Kruschke, J. K. (2008). Bayesian approaches to associative learning:
From passive to active learning. \emph{Learning \& Behavior},
\emph{36}(3), 210--226.

\leavevmode\vadjust pre{\hypertarget{ref-loh2022}{}}%
Loh, K., Fintor, E., Nolden, S., \& Fels, J. (2022). Children's
intentional switching of auditory selective attention in spatial and
noisy acoustic environments in comparison to adults. \emph{Developmental
Psychology}, \emph{58}(1), 69.

\leavevmode\vadjust pre{\hypertarget{ref-mcallister2019}{}}%
McAllister, A., Rantala, L., \& Jónsdóttir, V. I. (2019). The others are
too loud! Children's experiences and thoughts related to voice, noise,
and communication in nordic preschools. \emph{Frontiers in Psychology},
\emph{10}, 1954.

\leavevmode\vadjust pre{\hypertarget{ref-mcmillan2016}{}}%
McMillan, B. T., \& Saffran, J. R. (2016). Learning in complex
environments: The effects of background speech on early word learning.
\emph{Child Development}, \emph{87}(6), 1841--1855.

\leavevmode\vadjust pre{\hypertarget{ref-panfili2017}{}}%
Panfili, H., L. M. (n.d.). The UW/NU corpus. Version 2.0. Retrieved from
\url{https://depts.washington.edu/phonlab/projects/uwnu.php}

\leavevmode\vadjust pre{\hypertarget{ref-quine1960}{}}%
Quine, W., \& Van, O. (1960). Word and object: An inquiry into the
linguistic mechanisms of objective reference.

\leavevmode\vadjust pre{\hypertarget{ref-rakison2022}{}}%
Rakison, D. H. (2022). Fear learning in infancy: An evolutionary
developmental perspective. In \emph{Evolutionary perspectives on
infancy} (pp. 303--323). Springer.

\leavevmode\vadjust pre{\hypertarget{ref-ruggeri2019}{}}%
Ruggeri, A., Swaboda, N., Sim, Z. L., \& Gopnik, A. (2019). Shake it
baby, but only when needed: Preschoolers adapt their exploratory
strategies to the information structure of the task. \emph{Cognition},
\emph{193}, 104013.

\leavevmode\vadjust pre{\hypertarget{ref-saffran1996}{}}%
Saffran, J. R., Aslin, R. N., \& Newport, E. L. (1996). Statistical
learning by 8-month-old infants. \emph{Science}, \emph{274}(5294),
1926--1928.

\leavevmode\vadjust pre{\hypertarget{ref-settles2009}{}}%
Settles, B. (2009). Active learning literature survey.

\leavevmode\vadjust pre{\hypertarget{ref-shannon1948}{}}%
Shannon, C. E. (1948). A mathematical theory of communication. \emph{The
Bell System Technical Journal}, \emph{27}(3), 379--423.

\leavevmode\vadjust pre{\hypertarget{ref-simon2022}{}}%
Simon, K. R., Merz, E. C., He, X., \& Noble, K. G. (2022). Environmental
noise, brain structure, and language development in children.
\emph{Brain and Language}, \emph{229}, 105112.

\leavevmode\vadjust pre{\hypertarget{ref-sloutsky2017}{}}%
Sloutsky, V. M., Yim, H., Yao, X., \& Dennis, S. (2017). An associative
account of the development of word learning. \emph{Cognitive
Psychology}, \emph{97}, 1--30.

\leavevmode\vadjust pre{\hypertarget{ref-stahl2015}{}}%
Stahl, A. E., \& Feigenson, L. (2015). Observing the unexpected enhances
infants' learning and exploration. \emph{Science}, \emph{348}(6230),
91--94.

\leavevmode\vadjust pre{\hypertarget{ref-stevens2017}{}}%
Stevens, J. S., Gleitman, L. R., Trueswell, J. C., \& Yang, C. (2017).
The pursuit of word meanings. \emph{Cognitive Science}, \emph{41},
638--676.

\leavevmode\vadjust pre{\hypertarget{ref-thiele2021}{}}%
Thiele, M., Hepach, R., Michel, C., Gredebäck, G., \& Haun, D. B.
(2021). Social interaction targets enhance 13-month-old infants'
associative learning. \emph{Infancy}, \emph{26}(3), 409--422.

\leavevmode\vadjust pre{\hypertarget{ref-thomas2022}{}}%
Thomas, A. J., Saxe, R., \& Spelke, E. S. (2022). Infants infer
potential social partners by observing the interactions of their parent
with unknown others. \emph{Proceedings of the National Academy of
Sciences}, \emph{119}(32), e2121390119.

\leavevmode\vadjust pre{\hypertarget{ref-tosatto2022}{}}%
Tosatto, L., Fagot, J., Nemeth, D., \& Rey, A. (2022). The evolution of
chunks in sequence learning. \emph{Cognitive Science}, \emph{46}(4),
e13124.

\leavevmode\vadjust pre{\hypertarget{ref-wu2021}{}}%
Wu, Y., \& Gweon, H. (2021). Preschool-aged children jointly consider
others' emotional expressions and prior knowledge to decide when to
explore. \emph{Child Development}, \emph{92}(3), 862--870.

\end{CSLReferences}

\bibliographystyle{apacite}


\end{document}
