% Template for Cogsci submission with R Markdown

% Stuff changed from original Markdown PLOS Template
\documentclass[10pt, letterpaper]{article}

\usepackage{cogsci}
\usepackage{pslatex}
\usepackage{float}
\usepackage{caption}

% amsmath package, useful for mathematical formulas
\usepackage{amsmath}

% amssymb package, useful for mathematical symbols
\usepackage{amssymb}

% hyperref package, useful for hyperlinks
\usepackage{hyperref}

% graphicx package, useful for including eps and pdf graphics
% include graphics with the command \includegraphics
\usepackage{graphicx}

% Sweave(-like)
\usepackage{fancyvrb}
\DefineVerbatimEnvironment{Sinput}{Verbatim}{fontshape=sl}
\DefineVerbatimEnvironment{Soutput}{Verbatim}{}
\DefineVerbatimEnvironment{Scode}{Verbatim}{fontshape=sl}
\newenvironment{Schunk}{}{}
\DefineVerbatimEnvironment{Code}{Verbatim}{}
\DefineVerbatimEnvironment{CodeInput}{Verbatim}{fontshape=sl}
\DefineVerbatimEnvironment{CodeOutput}{Verbatim}{}
\newenvironment{CodeChunk}{}{}

% cite package, to clean up citations in the main text. Do not remove.
\usepackage{apacite}

% KM added 1/4/18 to allow control of blind submission
\cogscifinalcopy

\usepackage{color}

% Use doublespacing - comment out for single spacing
%\usepackage{setspace}
%\doublespacing


% % Text layout
% \topmargin 0.0cm
% \oddsidemargin 0.5cm
% \evensidemargin 0.5cm
% \textwidth 16cm
% \textheight 21cm

\title{Preschool children reason about third-party goals when evaluating
acoustic environments}


\author{{\large \bf Rondeline M. Williams (rondelinewilliams@stanford.edu)} \\ Department of Psychology, Stanford University \\ Stanford, CA 94305 USA \AND {\large \bf Michael C.~Frank (mcfrank@stanford.edu)} \\ Department of Psychology, Stanford University \\ Stanford, CA 94305 USA}

\newlength{\cslhangindent}
\setlength{\cslhangindent}{1.5em}
\newenvironment{CSLReferences}%
  {}%
  {\par}

\begin{document}

\maketitle

\begin{abstract}
Despite the unpredictable and ubiquitous nature of noise in the natural
acoustic environment, most children still manage to extract the
linguistic, cognitive, and social information needed to engage with
their world. This is no small feat. We examined what strategies children
use to navigate different acoustic environments. One possibility we test
is that children can select acoustic contexts that are consistent with
particular goals. In Experiment 1, we presented preschool children with
a set of auditory stimuli, meant to approximate various acoustic
environments, and activity goals to complete within those environments.
Children integrated auditory information with goals to select the best
environment. To assess the flexibility of children's decision-making,
Experiment 2 built on this framework by replacing familiar activity
goals with relatively less familiar ones. In preliminary data, adults
and preschoolers reliably evaluated acoustic environments that best
matched these less familiar activities, providing evidence for flexible
reasoning about goal-consistent environments.

\textbf{Keywords:}
active learning; associative learning; auditory noise; cognitive
development; decision making
\end{abstract}

\hypertarget{introduction}{%
\section{Introduction}\label{introduction}}

Children are excavators; they routinely build linguistic, cognitive,
social, and emotional skills through interacting with their
environments. They can adjust their attention to linguistic stimuli such
as grammar based on its present learnability (Gerken, Balcomb, \&
Minton, 2011). They can exploit the emotional expressions of others to
determine whether a novel object is worth exploration, thereby
maximizing efficiency (Wu \& Gweon, 2021). And when they do explore,
children are often accounting for both the structure of the environment
and their present goals to decide on an approach (Meder, Wu, Schulz, \&
Ruggeri, 2021). This flexibility in the learning system is highly
adaptive, as it offers a means for data extraction even in unfamiliar or
suboptimal learning conditions.

We can understand why children are such flexible learners across a
diverse range of environments through the lens of active learning. In
this account, children make decisions about what and how they learn,
contrasting with the more passive view that they merely absorb
information presented to them without an opportunity to make adjustments
(Raz \& Saxe, 2020). The active learning literature has typically
explored children's interactions with individual stimuli within the
environment (e.g., Settles, 2009). For example, previous work has shown
that preschool children use active learning strategies to approach
objects in a novel task in order to optimize performance (Ruggeri,
Swaboda, Sim, \& Gopnik, 2019). In this task, children either opened or
shook two sets of boxes, one of which contained an egg shaker. When
children were told that the egg shaker was equally likely to be found in
either set of boxes, they were more likely to shake the boxes first than
when they were told the shaker was more likely to be found in a
particular set of boxes. Even infants harness the utility of active
learning by updating their expectations about what could be learned from
an object that behaved unexpectedly, such as a ball moving through a
solid wall (Stahl \& Feigenson, 2015). Additionally, infants as young as
7 months have been shown to efficiently allocate their attention to
visual stimuli that are neither too complex nor too simple (Kidd,
Piantadosi, \& Aslin, 2012).

A traditional account of active learning considers how children engage
with individual stimuli within their environment to harness new
information. But more recent work has considered how children reason
about environmental supports for learning as well. This type of active
learning has been called ``ecological active learning'', and it requires
children to both identify features of their environment that are stable
and then adjust their exploration strategies to maximize learning within
this ecology (Ruggeri, 2022). Ecological active learning proposes that
the structure of the environment, and not merely the individual stimuli
within it, is critical for information-seeking.

Here we apply the ecological active learning perspective to children's
acoustic environment. Given that children with access to auditory input
can learn a great deal from their acoustic environment, is it also
possible that they can reason about how well their acoustic environment
supports particular goals? For example, a child might choose to read or
to be read to in a library because a quiet space best aligns with the
goal of taking in a storybook. We refer to this as environmental
selection.

Environmental selection is goal-directed: children integrate information
about their environment because they are motivated to achieve some
outcome. Children may not always be able to choose their environment,
however. But even if they cannot choose, children can engage in
activities that align with their current environment (moving and dancing
when music is on, for example), and they will exploit variation across
environments to achieve a range of goals that would have been less
efficient under a single set of conditions.

We focus here on acoustic environmental selection because children's
acoustic environments can have important downstream effects on learning
and development. Acoustic noise has serious implications for learning,
especially for young children. Children are notably worse than adults at
skills such as speech perception and word recognition in noise
(Bjorklund \& Harnishfeger, 1990; Klatte, Bergström, \& Lachmann, 2013),
and exhibit real challenges in word learning under background noise
constraints (McMillan \& Saffran, 2016). Because noise generally
increases cognitive load during certain attention and spatial tasks,
children are less able to flexibly adapt strategies to successfully
complete these tasks than adults (Loh, Fintor, Nolden, \& Fels, 2022).
There is also emerging evidence that high levels of sustained noise
exposure can lead to changes in cortical thickness in infants (Simon,
Merz, He, \& Noble, 2022). Importantly, effects of noise are not
happening exclusively at the unconscious level; even young children are
perceptually aware of excessive noise exposure (McAllister, Rantala, \&
Jónsdóttir, 2019). With this in mind, we consider whether environmental
selection could be an adaptive strategy for learning in noisy acoustic
environments.

In the current paper, we studied preschool children's environmental
selection. In Experiment 1, we asked children to match a set of goals to
auditory environments. Then in Experiment 2, to explore the conceptual
boundaries of this ability, we presented children with relatively less
familiar activities and asked them to complete the same task. Taken
together, this set of experiments aims to expand our understanding of
how children can exploit their acoustic environment for goal achievement
across a range of inputs.

\hypertarget{experiment-1}{%
\section{Experiment 1}\label{experiment-1}}

In our first experiment, we evaluated preschool children's environmental
selection, their integration of both auditory information and a third
party's goals, for familiar activities. We asked whether they would
differentially select environment-goal pairings that optimized another
person's goals. If children are systematically pairing based on
outcomes, this may suggest that they are, in fact, attuned to the
environment as a strategy to reach a set of goals.

\hypertarget{methods}{%
\subsection{Methods}\label{methods}}

\hypertarget{participants}{%
\subsubsection{Participants}\label{participants}}

72 children (3;0 - 5;11 years, mean age = 4.46 years , 4.2\% African
American/Black, 23.6\% Asian American/Pacific Islander, 27.8\%
Caucasian/White, 8.3\% Hispanic/Latinx, 26.4\% Multiracial, 8.3\% Other)
were recruited from either a local Bay Area nursery school or children's
museum. Participants were typically developing, had normal or
corrected-to-normal vision, and heard English at least 75\% of the time
at home. An additional 6 children were ultimately excluded from analysis
due to response bias (provided the same pattern of responses for 100\%
of trials), experimenter error, or severe lapses in attention. All
exclusion criteria were preregistered {[}osf.io/fm7wx{]}. Caregivers
provided written consent while children provided verbal assent before
participation.

\hypertarget{materials-and-procedure}{%
\subsubsection{Materials and Procedure}\label{materials-and-procedure}}

\begin{CodeChunk}
\begin{figure}[t]

{\centering \includegraphics{figs/e3-stimuli-1} 

}

\caption[Experimental setup and stimuli]{Experimental setup and stimuli. Participants were shown four wooden houses, each with an associated sound [instrumental music, multi-talker babble, silence, white noise], and a list of four activities [dance, read, sleep, talk] that two charactrs in the game wanted to complete. Participants determined whether the two characters should or should not complete an activity in each of the four houses. Responses were independent of each other.}\label{fig:e3-stimuli}
\end{figure}
\end{CodeChunk}

A trained undergraduate research assistant served as the experimenter
for the task. The experimenter first introduced participants to two
small plastic figures named Joe and Mandy and to four wooden houses with
a felt door on the front. The experimenter then showed participants a
list of four images, each depicting one activity Joe and Mandy wanted to
do together. The experimenter explained that when each door opened, a
sound would either play or or there would be silence. Joe and Mandy
could choose whether or not to complete an activity in each house, and
their decisions would be entirely based on participants' responses.
Importantly, these decisions were independent of each other;
participants could decide to have Joe and Mandy complete the same
activity in more than one room if appropriate. A sound button was
attached to the back of each house and hidden from the participant's
view so when the experimenter opened a door, they also pressed down on
the button to play the appropriate sound. The wooden houses were lined
up on a table several inches apart with the participant seated facing
the doors of the houses and the experimenter on the opposite side facing
the sound buttons. Figure 1 illustrates the setup, the four activities,
and the four auditory stimuli.

The experimenter began the task with the first image on the list and
told participants, ``It looks like Joe and Mandy want to {[}sleep{]}.
Let's look at each room and see if Joe and Mandy should {[}sleep{]}
inside.'' The experimenter then opened the door to the first house
{[}the experimenter always began with the first house on their left/the
first house on the participant's right{]} and pressed down on the sound
button. At the end of the audio clip, the experimenter closed the door
and asked participants two questions. Participants only heard each audio
clip once per trial. The experimenter repeated this process for the
three remaining houses before moving on to the next activity. In total,
participants completed 16 trials -- four trials for each activity times
four trials for each auditory stimulus. The presentation order of two
conditions was counterbalanced to create two additional conditions, for
a total of four conditions.

Each auditory stimulus was 7s in length and normalized to a root mean
square (RMS) amplitude of 65 dB. The multi-talker babble was an overlay
of five adult native English speakers reading short, unrelated sentences
(Panfili, Haywood, McCloy, Souza, \& Wright (2017)). The white noise was
engineered in Audacity. The instrumental music contained no human
speech. Both the activities and auditory stimuli were selected based on
a sample of adults run
previously.\footnote{In pilot testing, we noted that the auditory stimuli could suggest varying numbers of people inside the wooden houses. For example, the house paired with multi-talker babble might appear to have morepeople inside than the houses paired with instrumental music, silence, and white noise. This appearance might inadvertently influence children's decisions on whether or not a house is appropriate for a specific activity for reasons other than the auditory stimuli. To address this issue, we opened the top of each house and showed children that two other figures were inside.}

We asked participants two questions which served as our DVs: (1)
``Should Joe and Mandy {[}read/dance/sleep/talk{]} in this room?'' and
(2) ``Why did you say Joe and Mandy {[}should/shouldn't{]}
{[}read/dance/sleep/talk{]} in this room?''

\hypertarget{results-and-discussion}{%
\subsection{Results and Discussion}\label{results-and-discussion}}

\begin{CodeChunk}
\begin{figure*}[t]

{\centering \includegraphics{figs/e3b-bar-1} 

}

\caption[Results from Experiment 1]{Results from Experiment 1. Participants' rating of the appropriateness of an auditory stimulus and activity pairing. Individual bars correspond to one age bin of 3, 4, or 5. A rating score of 0 indicates a rejection of the pairing [Joe and Mandy should not complete a particular activity in this environment] while a score of 1 indicates an affirmation of the pairing [Joe and Mandy should complete a particular activity in this environment]. A 2-alternative forced choice design resulted in no preference at 50\%.  Error bars show 95\% confidence intervals.}\label{fig:e3b-bar}
\end{figure*}
\end{CodeChunk}

If preschool children can reason about how the acoustic environment
might influence goal achievement, and can make decisions to this end, we
should expect participants to show clear preferences for activities
paired with particular auditory stimuli, and indeed they appeared to.
Figure 2 depicts children's activity-auditory pairings by age.

We preregistered a Bayesian mixed-effects logistic regression from the
\texttt{rstanarm} package to predict participants' responses as a
function of auditory stimulus, activity, and age (centered), with a
maximal random effect structure (random intercept by participant)
(Goodrich, Gabry, Ali, \& Brilleman, 2020). In this and subsequent
models, we used the package default of weakly informative priors (normal
distributions on coefficients with SD=2.5, scaled to predictor
magnitudes).

Because we had four activities and four acoustic environments, there
were six main effects and 12 two-way interactions of activity and
environment (setting dance and music as the reference levels
respectively). All six main effects had negative coefficients (lower
levels of selection than music with dancing), and all 95\% CrIs did not
overlap zero. Importantly, all of the two-way interaction coefficients
were positive and all had 95\% CrIs not overlapping zero, indicating the
specificity of the relationship between activity and acoustic
environment.

There were some numerical developmental effects, but the coefficient on
age had a small estimated value and a CrI that overlapped with zero
(\(\beta = -0.01 [-0.09, 0.07]\)). The interaction between age and
multi-talker babble did have a substantial negative magnitude
(\(\beta = -0.2 [-0.34,-0.08]\)) indicating lower choice of that room
for older children. Numerically, even three-year-olds appeared to match
music to dancing and silence to sleeping, though their other preferences
were weaker.

In sum, these findings suggest that across the preschool years, children
can evaluate acoustic environments to make decisions about third-party
goals. Our results provide preliminary evidence that children as young
as three can engage in basic environmental selection, at least for
familiar activity pairs.

\hypertarget{experiment-2}{%
\section{Experiment 2}\label{experiment-2}}

In Experiment 1, we found that preschool children do engage in
environmental selection, such that they may make decisions about
environment utility for goal selection based on auditory information. We
also found that this ability was generally stable across the preschool
years. However, it is possible that children succeeded in this task not
because they were engaging in some cognitively flexible process, but
because they were relying on pure associations. For example, children
may have paired sleeping with silence because they typically sleep in
quiet environments, and not because they recognize that silence might be
the most optimal auditory environment for sleep.

One possibility, then, is that environmental selection is driven by
associative knowledge for young children, rather than by task- or
goal-based reasoning. Importantly, this more limited environmental
selection could still be useful -- after all, regardless of \emph{why}
you want quiet to sleep, this desire will still get you the same result.
But such associative links would allow for much less flexible
environment selection for learning activities in the face of acoustic
noise, so we were interested in whether children could perform the more
difficult task of finding an acoustic environment for activities with
varying degrees of novelty.

To test this question, we replaced the familiar activities in Experiment
1 with relatively less familiar ones. If children primarily reason about
the pairing between the acoustic environment and a set of goals through
pure association, they should have trouble pairing acoustic environments
with less familiar activities because they have had less exposure with
these pairings. If, however, children are actively updating information
about their environment and then using this information to inform new
goals, they should also succeed even when faced with goals of varying
novelty.

\hypertarget{methods-1}{%
\subsection{Methods}\label{methods-1}}

\hypertarget{participants-1}{%
\subsubsection{Participants}\label{participants-1}}

46 children (3;0 - 5;11 years, mean age = 4.51 years, 2.2\% African
American/Black, 34.8\% Asian American/Pacific Islander, 34.8\%
Caucasian/White, 4.3\% Hispanic/Latinx, 23.9\% Multiracial) were
recruited from either a local Bay Area nursery school or children's
museum. An additional 8 children were ultimately excluded from analysis.
This sample is a subset of the 72 children we intend to include in the
final preregistered sample.

\hypertarget{materials-and-procedure-1}{%
\subsubsection{Materials and
Procedure}\label{materials-and-procedure-1}}

The procedures for Experiment 2 were nearly identical to Experiment 1
with one notable difference. To determine whether preschool children use
environmental selection flexibly to relatively less familiar activities,
we presented participants with a new list of activities with varying
degrees of novelty- (1) Fraw: when someone reads you a bedtime story
right before you fall asleep, (2) Gobb: when you are looking for
something to do because you are really bored, (3) Plip: when you spin
around in circles to the beat until you get really dizzy, and (4) Terb:
when you don't want anyone else to know your tummy is making noise. We
selected these activities based on an adult sample we previously ran
online, where we found these four activities elicited the widest
distribution of responses among participants. The novelty of these four
activities fall along a gradient from something children may have
previously encountered (``fraw'' and ``gobb'') to something with which
they may have had less contact (``plip'' and ``terb''). While none of
the selected activities are purely novel, they all have novel labels and
may require children to consider multi-step actions as single behaviors.
For example, fraw is an action in which (1) someone is being read to and
(2) the context in which the story is read is at bedtime. This is
arguably different from pure reading despite the strong likelihood that
their optimal environments are the same. From this view, there may be
important utility in asking children to reason about these activities,
as they are reasonably distinct from those in Experiment 1.

We asked participants two questions which served as our DVs: (1)
``Should Joe and Mandy {[}fraw/gobb/plip/terb{]} in this room?'' and (2)
``Why did you say Joe and Mandy {[}should/shouldn't{]}
{[}fraw/gobb/plip/terb{]} in this room?''

\hypertarget{results-and-discussion-1}{%
\subsection{Results and Discussion}\label{results-and-discussion-1}}

\begin{CodeChunk}
\begin{figure*}[t]

{\centering \includegraphics{figs/e4-summary-fig-1} 

}

\caption[Results from (A) children and (B) adults in Experiment 2]{Results from (A) children and (B) adults in Experiment 2. While children made binary judgments, adults used a seven-point likert scale indicating complete match (7) to complete mismatch (1) between sounds and activities.}\label{fig:e4-summary-fig}
\end{figure*}
\end{CodeChunk}

If preschool children rely solely on association when evaluating the
acoustic environment, that is, they have acquired associative links
between familiar activities and their acoustic contexts, we should
expect that children will have no strong preferences for pairing less
familiar activities with any particular acoustic context. If, however,
children can reason flexibly about how the acoustic environment
influences goal optimization and outcomes, we should expect that
children show clear preferences for acoustic contexts even with
activities for which they have less experience.

Data collection for Experiment 2 is ongoing, so we present a preliminary
analysis of the data. Figure 3A shows the pattern of choices across all
participants (not disaggregated by age). Several of the activities
appear to show patterns of preference for one acoustic environment.

While we did not have sufficient power to fit our full preregistered
Bayesian mixed effects model, we did fit a subset model that did not
include effects of age. We set the reference level to be ``fraw'' (story
before bed) and silence. Critically, we found evidence for interactions
between two activities and either the music or babble category, such
that the music environment was preferred for ``plip'' (spinning to the
beat; \(\beta = 2.05 [0.67, 3.45]\)), and the babble environment for
``terb'' (hiding tummy noises; \(\beta = 1.97 [0.58, 3.06]\)).

These preliminary results give evidence that children appeared to be
reasoning about the goal of activities and how they fit with different
acoustic environments. Children may have drawn on familiar elements of
these new activities, including associations with bedtime stories or
spinning, but they were clearly reasoning in some way about these.
Perhaps the most interesting activity from this perspective was
``terbing'' (hiding tummy noises), where children would have to reason
that music might mask the sound of their tummy. We interpret the
``terbing'' result with caution, however, as this is likely to be a
challenging item and responses were relatively flat. Overall, these data
provide a first test of the idea that children are doing some kind of
reasoning beyond pure associative matching.

\hypertarget{adult-ratings-of-novel-activity-pairings}{%
\subsection{Adult Ratings of Novel Activity
Pairings}\label{adult-ratings-of-novel-activity-pairings}}

The previous findings suggest some differentiation in pairing auditory
stimuli with less familiar activities, but what pattern of results
should we expect? Given the novelty of the paradigm, we evaluated
results from our previous sample of adult participants and then compared
them to results from our sample of children.

\hypertarget{participants-2}{%
\subsubsection{Participants}\label{participants-2}}

37 adults (mean age = 40.43 years, 73\% Caucasian/White) were recruited
for an online study hosted on Prolific. An additional 2 participants
were ultimately excluded from analysis for failing one or both of the
attention checks.

\hypertarget{methods-and-procedure}{%
\subsubsection{Methods and Procedure}\label{methods-and-procedure}}

Participants completed a similar paradigm to children, but it was
adapted to an online, self-paced task. Participants watched animated
videos of a hallway exterior with one door centered in the middle of the
screen. When the door opened, one of four sounds played, followed by the
door closing, signaling the end of the video. Each video was 7s in
length and normalized to 65 dB RMS amplitude. Participants then
responded to the following prompt: ``I could {[}fraw/gobb/plip/terb{]}
in this room'', and were also provided with the definition of each.
Responses were collected via a likert rating scale from 1 (``not at all
well'') to 7 (``very well''). Each activity was presented one at a time,
and the presentation order was randomized across participants.

\hypertarget{results-and-discussion-2}{%
\subsubsection{Results and Discussion}\label{results-and-discussion-2}}

Figure 3B depicts adults' ratings for each activity by auditory
stimulus. Activity-sound interactions were similar between children and
adults. As with the children, we observed interactions between activity
and auditory stimulus such that adults also preferred pairing music with
``plip'' (spinning to the beat) (\(\beta = 3.04 [2.01, 4.03]\)).
However, adults preferred babble for ``terb'' while children did not
((\(\beta = 7.19 [6.18, 8.2]\)). Adult ratings for ``gobb'' (looking for
something to do) were relatively flat, as were children's judgments, but
children were less likely to pair ``gobb'' with white noise. These
findings seem to suggest children's reasoning about activity-auditory
pairings resulted in similar directional patterns to adults' ratings.

\hypertarget{general-discussion}{%
\section{General Discussion}\label{general-discussion}}

In this set of experiments, we explored preschool children's reasoning
about their acoustic environments and asked if they engage in
environmental selection to find environments that are consistent with
particular activities. In Experiment 1, we found that across the
preschool years, children can reliably evaluate the acoustic environment
to inform their decisions about third-party goals. In Experiment 2, we
asked whether children are primarily relying on associations or on
active learning when assessing activity feasibility by asking them to
reason about less familiar activities. Preliminary results show a trend
in children's flexibility on this task, such that they could reason
about activities for which they have less experience (with patterns of
ratings similar to those of adults). More importantly, they hint that
pure association may not accurately capture how children reason about
the optimality of their acoustic environment.

These findings support the notion that young children are attuned to
environmental features and can integrate this information for
decision-making related to achieving goals. That learning is situated in
an imperfect and often messy environment is all the more reason why
strategic exploration matters. If you can identify what is best learned
in a particular environment given the acoustic constraints, you might
both maximize your efficiency and reduce uncertainty, which bolsters
skill-building. Environmental selection offers a window into reasoning
about the acoustic context, and it highlights the value of active
learning in early childhood. Perhaps most advantageous, active learning
seems to be flexible, and supports children's exploration across a range
of experiences.

This work has some limitations, which motivate our future directions.
The preliminary findings, while not complete, are promising, and they
offer some ground for broaching our main research questions.
Additionally, this set of experiments explored children's reasoning
about the goals of others; it is possible that children's preferences
for certain acoustic environments may vary if they were instead asked to
reason about their own goals. In Experiment 2, we asked children to
reason about less familiar activities, but we are keenly aware that
these activities are not purely novel. Future work will explore
activities with greater degrees of novelty. This work also does not
directly test a strategy children might use to extract information from
the environment under noise constraints. Instead, it lays the foundation
for understanding how children evaluate acoustic environments with
varying degrees of noise, and a possible mechanism that drives it. By
exploring children's environmental evaluations and their flexibility
beyond familiar associations, we might later manipulate children's own
acoustic environments to observe the utility of environmental selection
in action. We believe the current studies are critical interim steps to
this end because they will inform the direction of future research.

This research also has potential utility in intervention efforts.
Environmental noise exposure is here to stay; noise pollution in the
United States affects everyone at some time or another, but some
evidence suggests that it disproportionately affects communities of
color and those of lower socioeconomic status, who tend to reside in
more densely populated regions (Casey et al., 2017). This could have
downstream consequences on linguistic and cognitive skills, as well as
on academic achievement. Future research should be sensitive to both the
acute and chronic effects of noise exposure on children, in particular,
and study strategies that can be implemented to ameliorate these
effects. We believe that environmental selection could be one such
strategy, and that by three years, children have the ability to use it
effectively.

\hypertarget{acknowledgements}{%
\section{Acknowledgements}\label{acknowledgements}}

We thank Ellen Markman for comments on the initial study design. We
thank Deba Elaiho, Renaecia Deleon Guerrero, Jason Miranda, Malia Perez,
and Karla Roman for data collection. We thank Bing Nursery School and
The San Jose Children's Discovery Museum for continued research support.

\hypertarget{references}{%
\section{References}\label{references}}

\setlength{\parindent}{-0.1in} 
\setlength{\leftskip}{0.125in}

\noindent

\hypertarget{refs}{}
\begin{CSLReferences}{1}{0}
\leavevmode\vadjust pre{\hypertarget{ref-bjorklund1990}{}}%
Bjorklund, D. F., \& Harnishfeger, K. K. (1990). The resources construct
in cognitive development: Diverse sources of evidence and a theory of
inefficient inhibition. \emph{Developmental Review}, \emph{10}(1),
48--71.

\leavevmode\vadjust pre{\hypertarget{ref-casey2017}{}}%
Casey, J. A., Morello-Frosch, R., Mennitt, D. J., Fristrup, K., Ogburn,
E. L., \& James, P. (2017). Race/ethnicity, socioeconomic status,
residential segregation, and spatial variation in noise exposure in the
contiguous united states. \emph{Environmental Health Perspectives},
\emph{125}(7), 077017.

\leavevmode\vadjust pre{\hypertarget{ref-gerken2011}{}}%
Gerken, L., Balcomb, F. K., \& Minton, J. L. (2011). Infants avoid
`labouring in vain'by attending more to learnable than unlearnable
linguistic patterns. \emph{Developmental Science}, \emph{14}(5),
972--979.

\leavevmode\vadjust pre{\hypertarget{ref-goodrich2020}{}}%
Goodrich, B., Gabry, J., Ali, I., \& Brilleman, S. (2020). Rstanarm:
{Bayesian} applied regression modeling via {Stan}. Retrieved from
\url{https://mc-stan.org/rstanarm}

\leavevmode\vadjust pre{\hypertarget{ref-kidd2012}{}}%
Kidd, C., Piantadosi, S. T., \& Aslin, R. N. (2012). The goldilocks
effect: Human infants allocate attention to visual sequences that are
neither too simple nor too complex. \emph{PloS One}, \emph{7}(5),
e36399.

\leavevmode\vadjust pre{\hypertarget{ref-klatte2013}{}}%
Klatte, M., Bergström, K., \& Lachmann, T. (2013). Does noise affect
learning? A short review on noise effects on cognitive performance in
children. \emph{Frontiers in Psychology}, \emph{4}, 578.

\leavevmode\vadjust pre{\hypertarget{ref-loh2022}{}}%
Loh, K., Fintor, E., Nolden, S., \& Fels, J. (2022). Children's
intentional switching of auditory selective attention in spatial and
noisy acoustic environments in comparison to adults. \emph{Developmental
Psychology}, \emph{58}(1), 69.

\leavevmode\vadjust pre{\hypertarget{ref-mcallister2019}{}}%
McAllister, A., Rantala, L., \& Jónsdóttir, V. I. (2019). The others are
too loud! Children's experiences and thoughts related to voice, noise,
and communication in nordic preschools. \emph{Frontiers in Psychology},
\emph{10}, 1954.

\leavevmode\vadjust pre{\hypertarget{ref-mcmillan2016}{}}%
McMillan, B. T., \& Saffran, J. R. (2016). Learning in complex
environments: The effects of background speech on early word learning.
\emph{Child Development}, \emph{87}(6), 1841--1855.

\leavevmode\vadjust pre{\hypertarget{ref-meder2021}{}}%
Meder, B., Wu, C. M., Schulz, E., \& Ruggeri, A. (2021). Development of
directed and random exploration in children. \emph{Developmental
Science}, \emph{24}(4), e13095.

\leavevmode\vadjust pre{\hypertarget{ref-panfili2017}{}}%
Panfili, L. M., Haywood, J., McCloy, D. R., Souza, P. E., \& Wright, R.
A. (2017). The UW/NU corpus. Version 2.0. Retrieved from
\url{https://depts.washington.edu/phonlab/projects/uwnu.php}

\leavevmode\vadjust pre{\hypertarget{ref-raz2020}{}}%
Raz, G., \& Saxe, R. (2020). Learning in infancy is active, endogenously
motivated, and depends on the prefrontal cortices.

\leavevmode\vadjust pre{\hypertarget{ref-ruggeri2022}{}}%
Ruggeri, A. (2022). An introduction to ecological active learning.
\emph{Current Directions in Psychological Science}, \emph{31}(6),
471--479.

\leavevmode\vadjust pre{\hypertarget{ref-ruggeri2019}{}}%
Ruggeri, A., Swaboda, N., Sim, Z. L., \& Gopnik, A. (2019). Shake it
baby, but only when needed: Preschoolers adapt their exploratory
strategies to the information structure of the task. \emph{Cognition},
\emph{193}, 104013.

\leavevmode\vadjust pre{\hypertarget{ref-settles2009}{}}%
Settles, B. (2009). Active learning literature survey.

\leavevmode\vadjust pre{\hypertarget{ref-simon2022}{}}%
Simon, K. R., Merz, E. C., He, X., \& Noble, K. G. (2022). Environmental
noise, brain structure, and language development in children.
\emph{Brain and Language}, \emph{229}, 105112.

\leavevmode\vadjust pre{\hypertarget{ref-stahl2015}{}}%
Stahl, A. E., \& Feigenson, L. (2015). Observing the unexpected enhances
infants' learning and exploration. \emph{Science}, \emph{348}(6230),
91--94.

\leavevmode\vadjust pre{\hypertarget{ref-wu2021}{}}%
Wu, Y., \& Gweon, H. (2021). Preschool-aged children jointly consider
others' emotional expressions and prior knowledge to decide when to
explore. \emph{Child Development}, \emph{92}(3), 862--870.

\end{CSLReferences}

\bibliographystyle{apacite}


\end{document}
